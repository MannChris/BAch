\subsection{$W_n$ (Räder)}
Der vorletzte Stop auf unserer Reise sind die sogenannten Wheel-Graphen. Hier wird zu einem zyklischen Graphen $C_n$ mit Knoten $\{v_1,..,v_n\}$, $n \geq 3$ ein weiterer Knoten $z$ hinzugefügt, der mit allen anderen Knoten benachbart ist, sodass der Wheel-Graph $W_{n}$ entsteht (Achtung: $W_n$ hat $n+1$ Knoten).
\begin{Tm}
Für die Anzahl der Spannbäume in einem Rad gilt:
\begin{equation}
 \mathit{k}(W_n) = (\frac{3+\sqrt{5}}{2})^n+(\frac{3+\sqrt{5}}{2})^n-2
 \label{wn}
\end{equation}
\end{Tm}
\textbf{Beweis:}\\
Um die Formel für die Berechnung der Anzahl der Spannbäume eines solchen Graphen herzuleiten, lassen wir von ~\cite{sedlacek_1970} inspirieren.
Wir beobachten, dass wir den Fan-Graphen $F_n$ bekommen, wenn wir die Kante $v_1v_n$ aus $W_n$ entfernen.
Die Anzahl der Spannbäume von $F_n$ kennen wir bereits von oben.
Um die Anzahl der Spannbäume von Rädern zu berechnen, zeigen wir zuerst die rekursive Beziehung
\begin{equation}
 \mathit{k}(W_{n+1}) = \mathit{k}(F_{n+1}) + \mathit{k}(F_n) + \mathit{k}(W_n)
\end{equation}
Um das zu tun, werden die Spannbäume von $W_{n+1}$ in drei verschiedene Klassen einteilen, wie man auch in den Abbildungen unten sehen kann:\\
\par %eingerückt
\begingroup
\leftskip=20pt% Parameter anpassen
\rightskip=20pt
\noindent
1) Alle Spannbäume, die die Kante $v_{n+1}v_1$, aber nicht die Kante $v_{n+1}z$ enthalten; das sind genau so viele, wie die Spannbäume von $W_n$.
\todo[inline, color=green]{Grafik dazu}
2)Alle Spannbäume, die die Kante $v_{n+1}v_1$ nicht enthalten; das sind genau so viele, wie die Spannbäume von $F_{n+1}$.\\
\todo[inline, color=green]{Grafik dazu}
3) Alle Spannbäume, die die Kante $v_{n+1}v_1$ und die Kante $v_{n+1}z$ enthalten; Wir beweisen im Folgenden, dass das so viele sind, wie die Spannbäume von $F_n$;
\par
\endgroup
Dafür werden wir zeigen, dass für die Anzahl der Spannbäume in Klasse 3 den gleichen rekursiven Formeln genügen wie die von $F_n$.\\
Sei $a_n$ die Anzahl der Subgraphen von $F_n$, die aus genau zwei Komponenten bestehen, von denen eine den Knoten $z$ und die andere $v_n$ enthält.
Wir definieren $b_n$ als die Anzahl der Spannbäume in Klasse 3, die die Kanten $v_nv_{n+1}$ und $v_nz$ nicht enthalten. 
Die nachfolgende Abbildung verdeutlicht, dass $\mathit{k}(F_{n+1})=2\mathit{k}(F_{n})+a_n$ für $n\geq 2$.
\todo[inline, color=green]{Grafik Konstruktion von Fn+1 aus Fn, und diesmal stimmt der Beweis wirklich}
\todo[inline]{Wenn die Grafik drin ist evtl noch ein-zwei Sätze dazu}
Sei $M_n$ die Menge der Spannbäume von $W_{n+1}$ aus Klasse 3;\\
Die nächste Grafik zeigt, dass $|M_{n+1}|=|M_n|+b_n$ ist.
\todo[inline, color=green]{Grafik zur Konstruktion, damit ist das offensichtlich}
\todo[inline]{Wenn die Grafik drin ist, evtl. noch ein-zwei Sätze dazu}
Wir sehen leicht, dass $\mathit{k}(F_2) = |M_2|$ und $a_2=b_2$; daraus schließen wir, dass die Anzahl der Spannbäume in Klasse 3 gleich $\mathit{k}(F_{n})$ ist, was wir zeigen wollten.
Da jeder Spannbaum von $W_{n+1}$ in genau einer der 3 Klassen ist, gilt die rekursive Beziehung
\begin{equation}
\mathit{k}(W_{n+1}) = \mathit{k}(F_{n+1}) + \mathit{k}(F_n) + \mathit{k}(W_n)
\label{eq:wrek}
\end{equation}
Wir werden nun den Beweis per Induktion über $n \in \mathbb{N}, \, n \geq 3$ vervollständigen, wobei uns natürlich zu Gute kommt, dass uns die Anzahl der Spannbäume von Fan-Graphen schon bekannt ist.\\
Für unseren Induktionsanfang sehen wir -zum Beispiel durch Anwendung von Kirchhoffs Matrix-Tree-Theorem- leicht, dass \begin{equation}
\mathit{k}(W_3) = 16 = (\frac{3+\sqrt{5}}{2})^3+(\frac{3+\sqrt{5}}{2})^3-2.
\end{equation}
Wir nehmen nun an, dass für ein $n \in \mathbb{N}$ die Formel 
\begin{equation}
 \mathit{k}(W_n) = (\frac{3+\sqrt{5}}{2})^n+(\frac{3+\sqrt{5}}{2})^n-2
\end{equation}
gilt.\\
Damit bleibt noch zu zeigen, dass
\begin{equation}
 \mathit{k}(W_{n+1}) = (\frac{3+\sqrt{5}}{2})^{n+1}+(\frac{3+\sqrt{5}}{2})^{n+1}-2.
\end{equation}
Das werden wir nun einfach ausrechnen.
Nachdem wir im vorherigen Kapitel herausgefunden haben, wieviele Spannbäume Fan-Graphen haben, setzen wir das und unsere Induktionsannahme in die Gleichung (\ref{eq:wrek}) ein, und erhalten:\\
\begin{equation}
\begin{aligned}
\mathit{k}(W_{n+1}) ={} & \frac{(3+\sqrt{5})^{n+1}-(3-\sqrt{5})^{n+1}}{2^{n+1}\sqrt{5}} + \frac{(3+\sqrt{5})^{n}-(3-\sqrt{5})^{n}}{2^{n}\sqrt{5}}\\
& + (\frac{3+\sqrt{5}}{2})^n+(\frac{3-\sqrt{5}}{2})^n-2
\end{aligned}
\end{equation}
Wir bringen fast alles auf einen Nenner, sortieren die Terme und bekommen
\begin{equation}
\begin{aligned}
\mathit{k}(W_{n+1}) = {}  & \frac{(3+\sqrt{5}+2+2\sqrt{5})(3+\sqrt{5})^{n}}{2^{n+1}\sqrt{5}} \\%% {} steht da nur, weils hin muss, wegen dem =
                        & -\frac{(3+\sqrt{5}+2-2\sqrt{5})(3-\sqrt{5})^{n}}{2^{n+1}\sqrt{5}}-2 
\end{aligned}
\end{equation}
\todo[inline]{zusammengehörige Terme farbig markieren}
Ausrechnen führt uns zu\\
\begin{equation}
\mathit{k}(W_{n+1}) = \frac{3+\sqrt{5}}{2})^{n+1}+(\frac{3+\sqrt{5}}{2})^{n+1}-2
\end{equation}
Damit ist unser Induktionsbeweis abgeschlossen und wir haben gezeigt, dass unser Satz \ref{wn} über die Anzahl der Spannbäume in einem Rad gilt.
\begin{flushright} $\Box$ \end{flushright} 
