\subsection{$W_n$ (Räder)}
Der vorletzte Stop auf unserer Reise sind die sogenannten Wheel-Graphen. Hier wird zu einem zyklischen Graphen $C_n$ mit Knoten $\{v_1,..,v_n\}$, $n \geq 3$ ein weiterer Knoten $z$ hinzugefügt, der mit allen anderen Knoten benachbart ist, sodass der Wheel-Graph $W_{n}$ entsteht (Achtung: $W_n$ hat $n+1$ Knoten).
\begin{Tm}
Für die Anzahl der Spannbäume in einem Rad gilt:
\begin{equation}
 \mathit{k}(W_n) = (\frac{3+\sqrt{5}}{2})^n+(\frac{3+\sqrt{5}}{2})^n-2
\end{equation}
\end{Tm}

\textbf{Beweis:}\\
Um die Formel für die Berechnung der Anzahl der Spannbäume eines solchen Graphen herzuleiten, lassen wir von ~\cite{sedlacek_1970} inspirieren.
Wir beobachten, dass wir den Fan-Graphen $F_n$ bekommen, wenn wir die Kante $v_1v_n$ aus $W_n$ entfernen.
Die Anzahl der Spannbäume von $F_n$ kennen wir bereits von oben.
Um die Anzahl der Spannbäume von Rädern zu berechnen, zeigen wir zuerst die rekursive Beziehung
\begin{equation}
 \mathit{k}(W_{n+1}) = \mathit{k}(F_{n+1}) + \mathit{k}(F_n) + \mathit{k}(W_n)
\end{equation}
Um das zu tun, werden die Spannbäume von $W_{n+1}$ in drei verschiedene Klassen einteilen, wie man auch in den Abbildungen unten sehen kann:\\
1) Alle Spannbäume, die die Kante $v_{n+1}v_1$, aber nicht die Kante $v_{n+1}z$ enthalten; das sind genau so viele, wie die Spannbäume von $W_n$. \\%%Grafik dazu einfügen
2)Alle Spannbäume, die die Kante $v_{n+1}v_1$ nicht enthalten; das sind genau so viele, wie die Spannbäume von $F_{n+1}$.\\%%Grafik dazu einfügen
3) Alle Spannbäume, die die Kante $v_{n+1}v_1$ und die Kante $v_{n+1}z$ enthalten; jetzt beweisen wir, dass das so viele sind, wie die Spannbäume von $F_n$.\\
Dafür werden wir zeigen, dass für die Anzahl der Spannbäume in Klasse $3$ den gleichen rekursiven Formeln genügen wie die von $F_n$.\\
\todo[inline, color=green]{Bilder davon}
Da jeder Spannbaum von $W_{n+1}$ in genau einer dieser Klassen ist, gilt die rekursive Beziehung
$\mathit{k}(W_{n+1}) = \mathit{k}(F_{n+1}) + \mathit{k}(F_n) + \mathit{k}(W_n)$\\
Wir werden nun den Beweis per Induktion über $n \in \mathbb{N}, \, n \geq 3$ vervollständigen, wobei uns natürlich zu Gute kommt, dass uns die Anzahl der Spannbäume von Fan-Graphen schon bekannt ist.\\
Für unseren Induktionsanfang sehen wir -zum Beispiel durch Anwendung von Krichhoffs Matrix-Tree-Theorem- leicht, dass $\mathit{k}(W_3) = 16 = (\frac{3+\sqrt{5}}{2})^3+(\frac{3+\sqrt{5}}{2})^3-2$.\\
Wir nehmen nun an, dass für ein $n \in \mathbb{N}$ die Formel $\mathit{k}(W_n) = (\frac{3+\sqrt{5}}{2})^n+(\frac{3+\sqrt{5}}{2})^n-2$ gilt.\\
Damit bleibt noch zu Zeigen, dass  $\mathit{k}(W_{n+1}) = (\frac{3+\sqrt{5}}{2})^{n+1}+(\frac{3+\sqrt{5}}{2})^{n+1}-2$.\\
Wie wir oben bereits gezeigt haben, gilt
$\mathit{k}(W_{n+1}) = \mathit{k}(F_{n+1}) + \mathit{k}(F_n) + \mathit{k}(W_n)$\\
Nachdem wir im vorherigen Kapitel herausgefunden haben, wieviele Spannbäume Fan-Graphen haben, setzen wir das und unsere Induktionsannahme sofort ein, und erhalten:\\
$\mathit{k}(W_{n+1}) = \frac{(3+\sqrt{5})^{n+1}-(3-\sqrt{5})^{n+1}}{2^{n+1}\sqrt{5}} + \frac{(3+\sqrt{5})^{n}-(3-\sqrt{5})^{n}}{2^{n}\sqrt{5}} + (\frac{3+\sqrt{5}}{2})^n+(\frac{3-\sqrt{5}}{2})^n-2$\\
Einfaches Umformen führt uns zu\\
\todo[inline]{Rechnung! und danach zusammengehörige Terme farbig markieren}
Damit haben wir erfolgreich gezeigt, dass für die Anzahl der Spannbäume in $W_n$ gilt:\\

\todo[inline, color=yellow]{Rechnungen evtl. in equations packen}

