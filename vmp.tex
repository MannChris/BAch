\subsection{vollständige multipartite Graphen}
 
Als nächste Graphenklasse betrachten wir vollständige multipartite Graphen.
\todo[inline]{vielleicht Notation}

\begin{Tms}
 Für die Anzahl der Spannbäume in einem vollständig m-partiten Graphen $K_{n_1,..,n_m}$ mit n Knoten gilt:\\
 $\mathit{k}(K_{n_1,..,n_m})=n^{m-2}\prod_{i=1}^{m}(n-n_1)^{n_i-1}$
\end{Tms}
\textbf{Beweis:}
An dieser Stelle beweisen wir wie in ~\cite{austin_1960}, dass\\ \\ \\%%blablabla
Damit ist auch unser Satz bewiesen, denn es ist ein äquivalentes Problem zu dem ebengenannten, da wir einfach die Knoten gleicher Farbe einfach als die Knoten einer Partition betrachten können.
In unserem Beweis werden wir zuerst.....
\todo[inline]{Beweis fertigmachen}
