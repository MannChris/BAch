\subsection{vollständige multipartite Graphen}
 
Als nächste Graphenklasse betrachten wir vollständige multipartite Graphen.
\todo[inline]{Notation}

\begin{Tms}
 Für die Anzahl der Spannbäume in einem vollständig m-partiten Graphen $K_{n_1,..,n_m}$ mit n Knoten gilt:\\
 $\mathit{k}(K_{n_1,..,n_m})=n^{m-2}\prod_{i=1}^{m}(n-n_1)^{n_i-1}$
\end{Tms}
\textbf{Beweis:}
Wir beweisen den Satz ähnlich wie Austin in ~\cite{austin_1960}, der ein äquivalentes Problem zu dem ebengenannten bewiesen hat.\\
Dazu werden wir im Geist dieser Arbeit Kirchhoffs Matrix-Tree-Theorem verwenden.\\
Zuerst werden wir bemerken, dass alle Laplace-Matrizen, die unseren Sachverhalt beschreiben, bei geschickter Nummerierung Blockmatrizen einer bestimmten Gestalt sind und Schlüsse über deren Kofaktoren  ziehen. Im nächsten Schritt werden wir dann einen beliebigen solchen Graphen auswählen und die entsprechenden Werte einsetzen. Mit Kirchhoffs Matrix-Tree-Theorem folgt dann der Satz.\\

\todo[inline]{Beweis fertigmachen}
