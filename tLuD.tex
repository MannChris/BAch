\section{Grundlegende Definitionen und Notationen}
Wir beginnen damit, ein paar wichtige Begriffe und Notationen einzuführen, die wir später häufiger benutzen werden.
In einem Matrix-Tree-Theorem wird immer ein Zusammenhang zwischen bestimmten Matrizen und den Spannbäumen eines Graphen beschrieben. Daher bieten sich die folgenden zwei Definitionen an, von denen wir eine für ungerichtete und die andere für gerichtete Graphen verwenden werden.
\begin{Df}\textbf{Laplacematrix}\\
Für einen Graphen $G=(V,E)$ indizieren wir die Zeilen und Spalten einer Matrix mit den Knoten von G. Die Matrix mit den Einträgen
$$
l_{ij}=
\begin{cases}
 d(i),\, \, falls\,\, i=j,\\
 (-1), \, \, falls \,\, ij \in E,\\
 0, \,\, sonst.
\end{cases}
$$
nennen wir die Laplacematrix $L(G)$ des Graphen $G$.
\end{Df}
Für einen Knoten $i$ bezeichnen wir mit $d^{-}(i)$ seinen Ausgangsgrad.
\begin{Df}\textbf{Kirchhoffmatrix}\\
 Für einen gerichteten Multigraphen $D$ mit Knoten $1,..,n$ definieren wir die Kirchhoffmatrix $K(D)$ wie folgt:\\
 $$
k_{ij}=
\begin{cases}
 d^{-}(i),\, \, falls\,\, i=j,\\
 (-m), \, \, wobei \,\, m \,\, die\, Anzahl\, der\, Kanten\, von\,\, i\,\, nach \,\,j\,\, ist.
\end{cases}
$$
Wobei wir wieder die Zeilen und Spalten mit den Knoten von $D$ indiziert haben.\\
Nachdem wir in Tuttes Matrix-Tree-Theorem eine Zeile und Spalte streichen müssen, bezeichnen wir mit $K_{\bar{i}}(D)$ die Matrix, die entsteht, wenn man aus $K(D)$ die $i$-te Zeile und Spalte löscht.
\end{Df}
Im Verlauf dieser Arbeit werden wir immer wieder die Anzahl der Spannbäume eines Graphen ausrechnen, daher verwenden wir $\mathit{k}(G)$ als die Anzahl der Spannbäume eines beliebigen Graphen $G$.
\todo[inline, color=yellow]{Stand jetzt werde ich die Definitionen als solche hervorheben, Notation jedoch im Fließtext unterbringen, passt das so?-JA}
