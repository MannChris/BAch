\subsection{Tuttes Matrix-Tree-Theorem}
Um die Version des Matrix-Tree-Theorems für gerichtete Multigraphen und den dazugehörigen Beweis zu verstehen, müssen wir zuerst den Begiff der Aboreszenz einführen.
Als Aboreszenz aus einem Knoten bezeichnen wir einen gerichteten Baum, der diesen Knoten als Wurzel hat, wobei die Kanten von der Wurzel ausgehen.\\
Nun können wir ein Matrix-Tree-Theorem für gerichtete Multigraphen formulieren:
\begin{Tms}[Tuttes Matrix-Tree-Theorem]
\sloppypar
Sei $\,D\,$ ein gerichteter Multigraph mit Kirchoffmatrix $\,K(D)$.\;  Die Anzahl der Aboreszenzen aus einem Knoten $\,i\,$ ist gleich der $\,\det(K_{\bar{i}}(D))$.\; 
\par
\end{Tms}
Um das zu beweisen lassen wir uns von \cite{bang-jensen_2009} inspirieren.
Also zeigen wir zuerst folgendes Lemma:
\begin{Lms}
Sei $\,D\,$ ein gerichteter Multigraph mit maximalem Eingangsgrad $\,=1\,$ und $\,i\,$ ein Knoten in $\,D$.\; \\
Dann hat $\,D\,$ maximal eine Aboreszenz mit Wurzel $\,i$.\; Des Weiteren ist $\,\det(K_{\bar{i}}(D)) \in \{0,1\}$
und genau dann, wenn $\,D\,$ eine Aboreszenz mit Wurzel $\,i\,$ besitzt, ist $\,\det(K_{\bar{i}}(D)) = 1$.\; 
\label{L1}
\end{Lms}
\textbf{Beweis von Lemma ~\ref{L1}:}\\
Wir nehmen zuerst an, dass $\,D\,$ eine Aboreszenz mit Wurzel $\,i\,$ hat.\\
Da der maximale Eingangsgrad $\,=1\,$ ist, schließen wir, dass für jeden Knoten außer $\,i$,\; die eine eingehnende Kante in dieser Aboreszenz enthalten ist. \\
Weil das Vertauschen von Zeilen einer Matrix deren Determinante nicht ändert, dürfen wir annehmen, dass $\,i=1\,$ ist und die übrigen Knoten in der Reihenfolge einer Breitensuche durchnummeriert sind.\\ 
Dann ist nämlich $\,K_{\bar{i}}(D)\,$ eine obere Dreiecksmatrix mit Diagonaleinträgen $\,=1$,\; also $\,\det(K_{\bar{i}}(D)) = 1$.\; 
Jetzt nehmen wir an, dass $\,D\,$ keine Aboreszenz mit Wurzel $\,i\,$ besitzt.\\
Falls ein anderer Knoten als $\,i\,$ Eingangsgrad $\,=0\,$ hat, sind die alle Einträge der entsprechenden Spalte von $\,K_{\bar{i}}(D)\,$ und damit auch $\,\det(K_{\bar{i}}(D))\,$ gleich $\,0$\\
Deshalb dürfen wir zu guter Letzt annehmen, dass für alle von $\,i\,$ verschiedenen Knoten der Eingangsgrad $\,=1\,$ ist.\\
Weil $\,D\,$ keine Aboreszenz mit Wurzel $\,i\,$ besitzt, hat $\,D\,$ einen Zyklus, der $\,i\,$ nicht enthält.\\
Da aber jeder Knoten $\,\neq i\,$ Eingangsgrad $\,=1\,$ hat, sind die Spalten, die mit den Knoten in diesem Zyklus korrespondieren, linear abhängig und damit $\,\det(K_{\bar{i}}(D)) = 0$.\; \\
Damit haben wir unser Lemma bewiesen.
\begin{flushright} $\,\Box\,$ \end{flushright} 

Nun können wir uns dem Beweis von Tuttes Matrix-Tree-Theorem widmen.

\textbf{Beweis von Tuttes Matrix-Tree-Theorem:}
Wir werden die Matrix $\,K(D)\,$ in kleinere Matrizen zerlegen und dann mit dem Lemma von oben arbeiten.\\
Bei der Zerlegung in kleinere Matrizen erinnern wir uns an ein Ergebnis aus der linearen Algebra:
Für eine Matrix aus Spalten $\,c_1,..,c_n\,$ mit jeweils $\,n\,$ Elementen gilt:
\begin{equation}
 \det(c_1,..,(c_i+\prime{c_i}),..,c_n) = \det(c_1,..,c_i,..,c_n) + \det(c_1,..,\prime{c_i},..,c_n)
\end{equation}
Weil sich die einzigen positiven Einträge von $\,K(D)\,$ auf der Diagonale befinden und die Spaltensummen alle $\,=0\,$ sind, können wir $\,K(D)\,$ auf diese Weise in  $\,s=\prod_{i=1}^nk_{ii}\,$ $\,n \times n$-Matrizen $\,K(D_i), i\in\{1,..,s\}\,$ zerlegen, indem wir jede Spalte als Summe von $\,k_{ii}\,$ Spalten schreiben von denen jede mit einer in $\,i\,$ eingehenden Kante korrespondiert.\\
Ohne Beschränkung der Allgemeinheit können wir $\,i=1\,$ annehmen.\\
Wir werden jetzt $\,\det(K_{\bar{1}}(D))\,$ mithilfe der Matrizen von oben vereinfachen und dann ausrechnen.\\
Hierzu sei $\,\hat{D}\,$ der Graph der aus $\,D\,$ durch löschen aller in den Knoten $\,1\,$ eingehenden Kanten entsteht und $\,d^{-}(j)\,$ der Eingangsgrad eines Knoten $\,j$.\;  Dann folgt:
\begin{equation}
 \det(K_{\bar{1}}(\hat{D})) = \sum_{e_2}^{d^{-}(2)}\det(K_{e_2}(\hat{D}))
\end{equation}
,wobei $\,K_{e_j}(\hat{D})\,$ aus $\,\hat{D}\,$ durch löschen aller in den Knoten $\,j\,$ eingehenden Kanten außer $\,e_j\,$ entsteht.\\
Wiederholen wir das für die übrigen Knoten, bekommen wir:
\begin{equation}
  \det(K_{\bar{1}}(\hat{D})) = \sum_{e_2}^{d^{-}(2)}...\sum_{e_n}^{d^{-}(n)}\det(K_{e_n})
\end{equation}
Mit Lemma~\ref{L1} schließen wir, dass genau das die Aboreszenzen mit Wurzel $\,i\,$ zählt.\\
Da $\,\det(K_{\bar{i}}(D))\,$ aus $\,K(D)\,$ durch löschen der mit Knoten $\,i\,$ korrespondierenden Zeile und Spalte entstanden ist und wir ohne Beschränkung der Allgemeinheit $\,i=1\,$ annehmen durften, gilt:
\begin{equation}
 \det(K_{\bar{1}}(\hat{D}))=\det(K_{\bar{1}}(D))=\det(K_{\bar{i}}(D))
\end{equation}
Das vervollständigt unseren Beweis von Tuttes Matrix-Tree-Theorem.
\begin{flushright} $\,\Box\,$ \end{flushright} 

