\subsection{Cartesische Produkte von Graphen}
In diesem Teil zeigen wir, was im Bezug auf die Anzahl der Spannbäume geschieht, wenn man das kartesische Produkt von Graphen bildet. %Wie schaut so ein Graph dann überhaupt aus? 
\todo[inline, color=yellow]{Ich werde wahrscheinlich ein/zwei Beispiele davon zeigen, z.B. Lattice-Graph, aber nicht mehr, weil das im Grund genommen einfach nur Rechnungen sind und das Matrix-Tree-Theorem nicht mehr als solches angewendet wird, sondern nur über den Satz unten(passt das?)}
\todo[inline, color=yellow]{Vielleicht ist es sinnvoll ein weiteres Kapitel mit einfachen Graphen wie Kreis-Graphen, Pfad-Graphen,etc. zu machen, dann könnte man sich in diesem Kapitel fast alle Rechnungen  ersparen und nur ein/zwei Beispiele geben, was man daraus "basteln" kann (Gute Idee?)}
\begin{Tms}
 Sei $G$ ein Graph mit $m$ Knoten und Eigenwerten $\mu_1(G),..,\mu_m(G)$ und $H$ ein Graph mit $n$ Knoten und Eigenwerten $\mu_1(H),..,\mu_n(H)$. \\
% Dann sind die Eigenwerte des kartesischen Produkts $G \times H$ genau $\mu_i(G)+\mu_j(H)$ mit $i \in \{ 1,..,m\}, j \in \{ 1,..,n\}$.
Dann hat der Graph $G \times H$ genau \\
$\displaystyle\prod_{i,j}(\mu_i(G)+\mu_j(H))1_{\{\mu_i(G)+\mu_j(H)\}\neq0}$\\ Spannbäume.
\end{Tms}

\todo[inline]{Beweis schreiben, Satz umformulieren zu einer Aussage über Spannbäume, Reminder:(Kronekersumme Aidb + idaB)}

\todo[inline]{Bei den Beispielen werden wir vollstänige Graphen nutzen, da muss noch kurz argumentiert werden wie da die Eigenwerte sind}
