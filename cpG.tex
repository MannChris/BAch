\subsection{Cartesische Produkte von Graphen}
In diesem Teil zeigen wir, was im Bezug auf die Anzahl der Spannbäume geschieht, wenn man das kartesische Produkt von Graphen bildet. %Wie schaut so ein Graph dann überhaupt aus? 
\begin{Lm}
eigenwerte kartesisches Produkt v. Graphen (Kronekersumme Aidb + idaB)
\todo[inline, color=red]{das ist eigentlich kein Lemma bleibt aber für den Moment so markiert}
\end{Lm}

\begin{Tms}
 Sei $G$ ein Graph mit $m$ Knoten und Eigenwerten $\mu_1(G),..,\mu_m(G)$ und $H$ ein Graph mit $n$ Knoten und Eigenwerten $\mu_1(H),..,\mu_n(H)$. \\
 Dann sind die Eigenwerte des kartesischen Produkts $G \times H$ genau $\mu_i(G)+\mu_j(H)$ mit $i \in \{ 1,..,m\}, j \in \{ 1,..,n\}$.
\end{Tms}
\todo[inline]{Beweis schreiben}

