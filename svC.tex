\graphicspath{{grafiken/}}

\subsection{Der vollständige Graph $K_n$ (Satz von Cayley)}

Als Einstieg soll der vollständige Graph mit n Knoten kurz $K_n$ dienen.\\%%%Wer hat das zuerst herausgefunden? Wo haben wir den Beweis her?Warum ist das interessant
\begin{Tms}[Satz von Cayley]
$K_n$ besitzt genau $n^{n-2}$ verschiedene Spannbäume.\\
\end{Tms}
\textbf{Beweis:}\\
Unser Beweis orientiert sich an ~\cite{Lau_2004}.
Wir wollen das Matrix-Tree-Theorem verwenden und betrachten deshalb die Determinante der Matrix $A_n\in M_{n-1}(\mathbb{Z})$, die durch das Streichen der ersten Zeile und Spalte der Laplacematrix $L_n(K_n)\in M_n(\mathbb{Z})$ entsteht:
\begin{equation}
A_n:=
\begin{pmatrix}
n-1&-1&\ldots&\ldots&\ldots&-1\\
-1&n-1&-1&\ldots&\ldots&-1\\
-1&-1&n-1&-1&\ldots&-1\\
\ldots&\ldots&\ldots&\ldots&\ldots&\ldots&\\
-1&\ldots&\ldots&\ldots&-1&n-1\\
\end{pmatrix}
\end{equation}
Da sich die Determinante durch elementare Zeilen- und Spaltenoperationen nicht ändert, dürfen wir die erste Spalte von allen anderen subtrahieren und erhalten:

\begin{equation}
det(A_n):=det
\begin{pmatrix}
n-1&-n&\ldots&\ldots&\ldots&-n\\
-1&n&0&\ldots&\ldots&0\\
-1&0&n&0&\ldots&0\\
\ldots&\ldots&\ldots&\ldots&\ldots&\ldots&\\
-1&0&\ldots&\ldots&0&n\\
\end{pmatrix}
\end{equation}
Mit demselben Argument wie oben addieren wir zur ersten Zeile alle übrigen und es ergibt sich:
\begin{equation}
det(A_n):=det
\begin{pmatrix}
1&0&\ldots&\ldots&\ldots&0\\
-1&n&0&\ldots&\ldots&0\\
-1&0&n&0&\ldots&0\\
\ldots&\ldots&\ldots&\ldots&\ldots&\ldots&\\
-1&0&\ldots&\ldots&0&n\\
\end{pmatrix}
\end{equation}
Wir berechnen den Wert dieser Determinante durch Entwicklung nach der ersten Zeile. Weil die Matrix $A_n$ eine $n-1 \times n-1$ Matrix ist, gilt:

\begin{equation}
 det(A_n)=n^{n-2}
\end{equation}
Nach Kirchhoff's Matrix-Tree-Theorem ist genau das die Anzahl der Spannbäume des $K_n$.
\begin{flushright} $\Box$ \end{flushright} 
