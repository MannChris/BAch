\graphicspath{{grafiken/}}

\subsection{Kirchhoffs Matrix-Tree-Theorem}
Nun werden wir das Matrix-Tree-Theorem für ungerichtete Graphen formulieren und beweisen, dass wir auch im weiteren Verlauf dieser Arbeit verwenden werden um die Anzahl der Spannbäume für verschiedene Graphenklassen zu bestimmen.
\begin{Tms}[Kirchoffs Matrix Tree Theorem]
Sei $G$ ein ungerichteter Graph und $L_n$  die dazugehörige Laplacematrix. 
Dann gilt:
% Bis hier Text ohne Einrückung
\par
\begingroup
\leftskip=20pt% Parameter anpassen
\rightskip=20pt
\noindent %ab hier der Text, der eingerückt werden soll
(1) Die Anzahl der Spannbäume von $G$ gleich einem beliebigen Kofaktor von $L_n$.\\
(2) Die Anzahl der Spannbäume von $G$ ist gleich $\frac{1}{n}\lambda_1\ldots\lambda_{n-1}$, wobei $\lambda_1,\ldots,\lambda_{n-1}$ die Eigenwerte von $L_n$ sind, die ungleich null sind.
\par
\endgroup
% ab hier wieder Text ohne Einrückung
\end{Tms}
\textbf{Beweis:}\\
Teil 1) des Kirchhoffs Matrix-Tree-Theorem folgt quasi direkt aus Tuttes Matrix-Tree-Theorem. \\
Sei $\vec{G}$ der gerichtete Graph, der entsteht, wenn man jede Kante in $G$ als zwei gerichtete ansieht.\\
Wir betrachten einen beliebigen Knoten aus $\vec{G}$, der natürlich auch in $G$ ist. \\
Da nach Definition jeder Knoten in jedem Spannbaum mit jedem anderen wegverbunden ist, korrespondiert jeder Spannbaum von $G$ mit genau einem out-branching aus unserem Knoten in $\vec{G}$. \\
Da jede Kante in $\vec{G}$ auch in die entgegengesetzte Richtung vorhanden ist, können wir schließen, dass $L_n=K(\vec{G})$, wobei $L_n$ die Laplacematrix von $G$ ist. \\
Jeder Kofaktor von $L_n$ ist also gleich jedem Kofaktor von $K(\vec{G})$.
\todo[inline]{Beweis: es ist irrelevant, welchen Kofaktor vo Ln wir nehmen!}
Wir folgern daraus mit Tuttes Matrix-Tree-Theorem, dass die Anzahl der Spannbäume in G gleich einem beliebigen Kofaktor von $L_n$ ist.\\ \\
Um Teil 2) zu zeigen, berufen wir uns auf ein bekanntes Ergebnis der linearen Algebra; \\
Das Produkt der Eigenwerte einer Matrix ist gleich der Summe seiner Hauptminoren. Das kann man zum Beispiel in ~\cite{meyer_2005} nachlesen. \\
Da $L_n$ $n$ Hauptminoren hat, folgt mit Teil 1), dass die Anzahl der Spannbäume von $G$ ist gleich $\frac{1}{n}\lambda_1\ldots\lambda_{n-1}$, wobei $\lambda_1,\ldots,\lambda_{n-1}$ die Eigenwerte von $L_n$ sind, die ungleich null sind. \\
Damit ist Kirchhoffs Matrix-Tree-Theorem bewiesen.
\todo[inline]{out-branching ersetzen}
 
