\graphicspath{{grafiken/}}

\subsection{Kirchhoff's Matrix-Tree-Theorem}
\begin{Tms}[Kirchoff ’s Matrix Tree Theorem]
Sei G ein ungerichteter Graph und $L_n$  die dazugehörige Laplacematrix. 
Dann gilt:
% Bis hier Text ohne Einrückung
\par
\begingroup
\leftskip=20pt% Parameter anpassen
\rightskip=20pt
\noindent %ab hier der Text, der eingerückt werden soll
(1) Die Anzahl der Spannbäume von G gleich einem beliebigen Kofaktor von $L_n$.\\
(2) Die Anzahl der Spannbäume von G ist gleich $\frac{1}{n}\lambda_1\ldots\lambda_{n-1}$, wobei $\lambda_1,\ldots,\lambda_{n-1}$ die Eigenwerte von $L_n$ sind, die ungleich null sind.
\par
\endgroup
% ab hier wieder Text ohne Einrückung
\end{Tms}

\begin{Lm}
 produkt der eigenwerte = summe der Hauptminoren
 \todo[inline, color=red]{das soll eigentlich kein Lemma sein, sondern ein well known fact, bleibt aber für den Moment so markiert}
\end{Lm}

\textbf{Beweis:}
\todo[inline]{Beweis aufschreiben, ca. 0.5d, leicht und Korollar aus Tuttes MTT}
 
