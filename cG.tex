\subsection{circulant Graphs}
%Als Letztes werden wir Graphen betrachten, deren Adjazenzmatrizen zyklisch sind.\\
%%Wo treten die auf (z.B. Circulant Graphs sind Cayley-Graphen zyklischer Gruppen)
Wir nennen einen Graphen circulant mit $n$ Knoten, wenn für $n \in \mathbb{N}$ und eine Menge $I \subset{\{1,..,\lfloor \frac{n}{2} \rfloor \}}\subset{\mathbb{N}}$ gilt, dass jeder Knoten $v$ genau zu jedem Knoten $(v+i) (\mod{n})$ mit $i \in I$ benachbart ist; wir bezeichnen solch einen Graphen kurz mit $C_n^I$.\\
Wir erinnern uns, dass eine $n\times n$ Matrix zyklisch genannt wird, falls jede Spalte aus der vorherigen durch Anwendung der Permutation $(1...n)$ hervorgeht.
Das ist bei den Adjazenzmatrizen unserer circulant Graphs, aufgrund dessen, wann Konten benachbart sind, natürlich der Fall.
Zu Gute kommt uns das bei der Berechnung der Anzahl von Spannbäumen in circulant Graphs, denn die Eigenwerte einer zyklischen Matrix sind wohlbekannt.%%gibts das Wort überhaupt?
Um die Formel für die Anzahl der Spannbäume überhaupt zu verstehen, müssen wir einen weiteren Begriff einführen.%%Grad von C_n^I erklären, (Grad maximalgrad eines knoten?)
Nachdem wir nun alles beisammen haben, formulieren wir folgenden Satz:

\begin{Tms}
Für die Anzahl der Spannbäume in circulant Graphs von Grad d gilt:\\
\begin{equation}
\mathit{k}\left( C_n^I \right) = \frac{1}{n} \prod_{j=1}^{n-1} \left(4 \sum_{i \in I} \sin^2 \left( \frac{ij\pi}{n}\right) \right),\,falls\,d\,gerade\,ist
\end{equation}
\begin{equation}
\mathit{k}\left( C_n^I \right) = \frac{1}{n} \prod_{j=1}^{n-1} \left(4 \sum_{i \in I} \sin^2 \left( \frac{ij\pi}{n}\right)-(-1)^j+1\right),\,falls\,d\,ungerade\,ist
\end{equation}
\end{Tms}

\textbf{Beweis:}\\
Wir beweisen den Satz wie ~\cite{wang_yang_1984}.\\


\begin{Bsps}[$C_n^2$ - Das Quadrat eines Kreises]

%evtl einen anderen Beweis für $C_n^{(1,2)}$ raussuchen, der jetzige ist lang
 %Bild von einem Square of a cycle
\end{Bsps} 
