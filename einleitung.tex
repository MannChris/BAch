\section{Einleitung}
\todo[inline]{Kirchhoff als "Erfinder" des MTT} 
\todo[inline]{Anwendungsgebiete des MTT außerhalb der Mathematik} 
\todo[inline]{Ausblick auf die Bachelorarbeit}
\todo[inline, color=pink]{Einleitung schreibe ich zuletzt}
1847 schrieb Kirchhoff in seiner Arbeit "Ueber[sic!] die Auflösung der Gleichungen, auf welche man bei der Untersuchung der linearen Verteilung galvanischer Ströme geführt wird;"~\cite{kirchhoff_1847} als erster implizit über den Zusammenhang zwischen Matrizen und der Anzahl von Spannbäumen von Graphen - explizit über Systeme von beliebig miteinander verbundenen Drähten und Gleichungen für die elektrischen Stromstärken in diesem System.\\
In der Graphentheorie sind Matrix-Tree-Theorems neben anderen Methoden, wie Prüfer-codes, beliebte Werkzeuge um die Anzahl der Spannbäume eines Graphen zu ermitteln.
Das wohl berühmteste, nach Kirchhoff benannte Matrix-Tree-Theorem wird oft als das Matrix-Tree-Theorem bezeichnet, wobei es auch noch andere Versionen, wie Tuttes Matrix-Tree-Theorem für gerichtete Multigraphen, dessen Beweis auch Teil dieser Arbeit sein wird, gibt.
Aber auch außerhalb der reinen Mathematik und der Theorie über elektrische Schaltkreise finden Matrix-Tree-Theorems ihre Anwendung.\\
In der Chemie gibt es einen Zusammenhang zwischen Spannbäumen und...~\cite{hinchliffe_2007}\\
Auch in der Informatik kann man Matrix-Tree-Theorems zum Beispiel dafür benutzen, die Anzahl von Spannbäumen von Netzwerken -die man als Graphen betrachen kann- zu berechnen, um dann Rückschlüsse über die Stabilität dieser Netzwerke zu ziehen...~\cite{yakoubi_2019}\\
In der Quantenphysik...~\cite{giovannetti_severini_2013}\\
Das Anwendungsspektrum von Matrix-Tree-Theorems ist also vielseitig.\\
In dieser Arbeit werden wir uns zwei Matrix-Tree-Theorems erschließen und damit im Anschluss die Anzahl der Spannbäume von Graphen einiger Klassen bestimmen.
