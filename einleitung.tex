\graphicspath{{grafiken/}}

\section{Einleitung}
\todo[inline]{Kirchhoff als "Erfinder" des MTT} 
\todo[inline]{Anwendungsgebiete des MTT außerhalb der Mathematik} 
\todo[inline]{Ausblick auf die Bachelorarbeit}

\todo[inline, color=pink]{Einleitung schreibe ich zuletzt}


Er fand heraus, dass ein Zusammenhang zwischen einer speziellen Matrix und der Anzahl der Spannbäume eines Graphen besteht. \\
Über diesen Zusammenhang macht ein Matrix-Tree-Theorem eine Aussage.\\
Auch außerhalb der reinen Mathematik und der Theorie über elektrische Schaltkreise finden Matrix-Tree-Theorems eine Anwendung.\\
In der Chemie gibt es einen Zusammenhang zwischen Spannbäumen und...\\
Auch in der Informatik kann man Matrix-Tree-Theorems zum Beispiel dafür benutzen, die Anzahl von Spannbäumen von Netzwerken -die man als Graphen betrachen kann- zu berechnen, um dann Rückschlüsse über die Stabilität dieser Netzwerke zu ziehen.\\
In der Quantenphysik...\\
Das Anwendungsspektrum von Matrix-Tree-Theorems ist also vielseitig.\\
In dieser Arbeit werden wir uns zwei Matrix-Tree-Theorems erschließen und damit danach die Anzahl der Spannbäume von Graphen einiger Klassen bestimmen.
